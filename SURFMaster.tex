\documentclass{article}
\usepackage{tabularx}
\usepackage{amsmath}
\usepackage{amssymb}
\usepackage{tikz}
\usetikzlibrary{timeline}
\usepackage{booktabs}
\usepackage{float}
\restylefloat{table}
\graphicspath{ {../} }
\usepackage[margin={1.5cm,2cm}]{geometry}
\usepackage{multicol}
\setlength\columnsep{1.5cm}
\usepackage{tabto}
\usepackage{pdflscape}
\usepackage{graphicx}
\usepackage{array}

\begin{document}
\begin{titlepage}
	\centering
	\begin{figure}[H]
	\centering
	%\includegraphics[scale=0.5]{logo_nasa_trio_black@2x.png}
	\end{figure}
	\vspace{2cm}
	{\scshape\LARGE NCRA-TIFR Project Proposal \par}
	\vspace{2cm}
	{\huge\bfseries \par}
	\vspace{2cm}
	{\Large\itshape Archit Sakhadeo\par}
	{\Large\itshape Rathin Desai\par}
	{\Large\itshape Shadab Shaikh\par}
	{\Large\itshape Shubhankar Deshpande\par}
	\vfill
	Mentor\par
	Dr. Yogesh  \textsc{Wadadekar}
	\par
	Dr. C. H. Ishwar \textsc{Chandra}

	\vfill

\end{titlepage}
\begin{multicols*}{2}
\section{Introduction}
\subsection{Morphological Classes of Radio Galaxies}

Radio galaxies with active nuclei can be distinguished based on their radio luminosity or brightness of their radio emissions in relation to their hosting environment. Some of the basic morphological classifications include point sources, extended sources i.e. sources with extended contours, double radio sources, jets, and lobes.


\subsection{Problems faced with current classification}

Currently Radio astronomers manually classify galaxies based on visual inspection of the images which is a slow procedure, and increases the time to production of scientific results. Further, it introduces uncertainities in the classification procedure, both of which are problems which can potentially be mitigated by using an automated approach.

Contemporary algorithms classify radio sources into at most three different classes. Our aim is to build a robust model capable of handling more than 2 classes.

\section{Objective}

\begin{itemize}
	\item Potentially discovering rare forms of radio sources by classification in different classes.
	\item Reduction in time to generate scientific results by radio astronomers.
	\item Deeper insight into topological representation of radio data during classification.
\end{itemize}

\section{Approach}

\subsection{Source Modelling}
  The first step would be source extraction using the standard technique of gaussian modelling. We propose to do this using the robust PyBDSM pipeline used for fitting gaussian distributions to radio sources. The software contains a plethora of features, from which we would be using a small subset. This would mainly include:

\begin{enumerate}
\item Source extraction using gaussian modelling of radio data.
\item Generation of a catalog file containing details of radio sources (RA, DEC, Size of Gaussian (min, max), etc.)
\end{enumerate}

\subsection{Cutout Generation}
The second step would be to convert the RA(Right Ascension) and DEC (Declination) values generated from the catalog, to their corresponding pixel values in the original image. Based on these pixel values we generate 10*10 px cutouts using as reference the co-ordinates of the center of the radio source. This involves a multistep procedure briefly including:
\begin{enumerate}
\item Using the astropy module to read the FITS image in the form of a matrix
\item Parsing through the generated catalog file using Pandas, and extracting data for each radio source such as RA, DEC, etc.
\item Converting the RA, DEC values using in-built functions in astropy to convert from WCS to pixel values.
\item Processing pixel values to account for difference in addressing between FORTRAN and C family of languages.
\item Slicing the image matrix assuming the reference pixel co-ordinates as the center of the source.
\item Storing the generated image cutout in a standard image format (JPEG), with name as (RA,DEC) values. 
\end{enumerate}

Prototype code for section \textit{3.1} and section \textit{3.2} has been written mainly for testing purposes. We used a sample image from the TGSS survey which was then processed using the first two steps of our pipeline to generate 470 cutout images.

More details may be found at: https://github.com/NCRA-TIFR/radiogen

\subsection{Data Preprocessing}

\begin{itemize}
\item Image Processing techniques
\end{itemize}

\subsection{Analytical Method}

Two broad steps that we plan to use: 
\begin{itemize}
	\item Statistical modelling of data to manually extract features. We plan to employ Scale-invariant feature transform (SIFT) algorithm to detect the features.  
	\item Classification of the radio galaxies based on these extracted features. Possible approaches : Naive Bayes, SVM and Random Forests.
\end{itemize}


\subsection{Empirical Approach}

We plan to deploy a Convolutional Neural Network model for classification which reduces the manual feature engineering part, and has achieved significant successes in object recognition and image classification tasks. ( Give References to papers) 
    


\section{How we predict it will solve the problem}

\section{Approximate timeline of completing the whole project}
\begin{itemize}
	\item 26th April to 11th May - literature survey
	\item Mid-August to October - working on basic prototype model individually by trying out multiple approaches
	\item October to November - choosing the approaches which work, and implementing them on all data, validation of the results.
	\item November to December - Refining the system, cleaning and commenting the code
\end{itemize} 





\section{Data used}
    TIFR GMRT Sky Survey data shall be used along with the data processed from GMRT cycle 20. We plan to use the data from other cycles once it has been run through our SPAM pipeline. 

\end{multicols*}
\end{document}
